\documentclass{amsart}
% \usepackage[lite]{amsrefs}
% \usepackage{amssymb}
\usepackage{graphicx}

\begin{document}

\title{A Super--quasicrystal of Penrose Tilings with $C_1$ Symmetry}
\author[MR Goodman]{Matthew Goodman}
\address{3Scan \\ 2122 Bryant \\ San Francisco CA, 94110}
\email{matt@3scan.com}

\date{\today}

\begin{abstract}
	abstract Stuffs here
\end{abstract}

\maketitle

\section{Introduction}
The study of aperiodic tilings has anchient origins beginning with Johannes Keppler's (incomplete) Aa
pattern. TODO CITEME  This work languished in relative obscurity until the work of Roger Penrose\cite{penrose1979pentaplexity}
found a set of matching rules which allowed this pattern to extend outward 
indefinately. Three emboidments of this pattern were cited in the initial patent application by Penrose\cite{penrose1979set},
but subsequent work revealed two of these patterns as duals.

These tilings were further analyzed and placed in algebraic frame by the work of 
Nicolaas Govert de Bruijn, which established both mechanisims for generation of the 
original Penrose tilings, as well as an infinite number of other embodiments.\cite{de1981algebraic1, de1981algebraic2}

These patterns would later be shown to be "quasicrystals" as they generate coherent Bragg 
diffcationion patterns under.  This means that they have a form of short range order than can
be seen in Fourier space.\cite{de1986quasicrystals} 

\section{Background Information}
We will start by reviewing the computational steps needed to be undertaken to generate a Penrose Tiling
with de Bruijn's methods.  These are somewhat complex, and to the knowledge of the author the supporting
software package represents the only open-source modular implementation.

The overall steps are as follows and will be broken down in increased detail in subsequent subsections:
\begin{enumerate}
  \item Select the magnitude of 5 bias vectors with specific restrictions.
  \item Establish a 2d isometric projection of the host 5d coordinate system.
  \item Draw $n$ gridlines perpendicular to each of these basis
  \item Compute the $10n^2$ intersections of these gridlines.
  \item Inspect each intersection and deteermine the tile shape and location to be drawn
\end{enumerate}
\subsection{Bias Vector Selection}
The selection of the bias vector determines in totality the form of a particular Penrose Tiling.
There is an important restriction that must be obeyed in the selection of these terms, namely that
the quotient of any two terms ($m$, $n$) must be an irrational number. 

(MRG NOTE: Irrational or non-real? this can also be expressed with a multaplicative identity.)

The original bias vector for Penrose's first tiling is given by:
\begin{equation}
  b_{p1}
  =
  \left[
     1, 
     \frac{\sqrt{5} - 1}{2},
     \frac{\sqrt{5} + 1}{2},
    -\frac{\sqrt{5} - 1}{2},
    -\frac{\sqrt{5} + 1}{2}
  \right]
\end{equation}
(MRG NOTE: Can I cite my own Stackoverflow question on this? I recall there being a helpful thesis link...)

Another simple bias vector we will use later is given by successive square roots of prime numbers:

\begin{equation}
 \left[1, \sqrt{2}, \sqrt{3}, \sqrt{5}, \sqrt{7}\right]
 \label{eq:punt}
\end{equation}



\subsection{}
There are TODO large steps to this 
There are  that Penrose tiles were two dimensional ``shadows'' of
a 5 dimensional lautice which:
\begin{enumerate}
  \item is projected from an isometric viewpoint -- where each axis entertains equal meter 
  as each other following projection projected into the 2d space
  \item The origin of the grid within the host coordinate, when projected into 2d, has offset such that no two vector
  components entertain a rational ratio. Otherwise state  of the grid is shifted in with a basis that obeys that form independent
  basis under the operation of integer multiplication? (Ackward wording here.) 
\end{enumerate}

The bias vector associated with the first Penrose tiling, P1 is as follows:

A visual representation of this form is seen in Figure \ref{fig:pentagrid-basis}.

\begin{figure}[h]
  \centering
  \includegraphics[width=0.4\linewidth]{figures/placeholder}
  \includegraphics[width=0.4\linewidth]{figures/placeholder}
  \caption{Left: Pentagrid basis vectors. To be thought of as a corner of a 5-dimensional box pointing away from the viewer.  Right: Pentagrid Lines coloring keyed to basis vectors from the left hand figure.}
  \label{fig:pentagrid-basis}
\end{figure}



\section{Background Information}

Penrose tiles are the opposite. They maintain the confusing position of looking predictable close-up, but 
from a suitable distance, have no order. They are an embodiment of irrationality rendered in 2d form. 
Much like pi or e they contain the works of Shakespere somewhere yet unseen if only the viewer looked deeply enough. Here are a couple of 
examples of renderings of them:
These are a recent discovery. Their discoverer, Roger Penrose is a brilliant mathematical mind, and 
even better illustrator of mathematical concepts. The first of these P1 was discovered “following Kepler’s method,”
which believe is math-code for doodled on a cocktail napkin. It is very striking in its original form:


Recently the understanding of Penrose Tilings has been greatly enhanced. Work in 1981 by a mathematician 
named Nicolaas Govert de Bruijn revealed two big results. First, he revealed that there are an infinite
number of Penrose tilings. This was unexpected, especially so when contrasted with the 17 regular tilings.
The second major discovery was that these tilings could be generated from the most regular geometric structure
out there, a grid, provided it lived in 5 dimensions and was viewed at the correct angle. Neither one of these
results are particularly obvious, so lets break them up a little bit, and talk through them.

Now, the important question I think we all have here is WTF does it mean to be a 5 dimensional grid? 
Well, that tricky quantity, we have some sense of stacked sugar cubes making a 3 dimensional grid, but as we seek 
to pull them out in the “next” dimension, we discover we have already used the 3 we know how to visualize. Tricky. 

We do however know that the pattern we are forming lives in the 2d plane. If you look at a cube straight at one 
of its 6 faces its ``shadow'' will form a square. This is true also for a 5d-cube, just from 80 directions instead. 

Ouch.

The particular we need to cast our shadow to get a Penrose tiling is thankfullly easy to motivate. In 3d,
it is starting straight into the box corner, or isometric, meaning a step in any axis will take us the same
distance in our shadow space. The 3d and 5d coordinate system’s axis shadow look like the following:

\begin{figure}[h]
  \centering
  \includegraphics[width=0.8\linewidth]{figures/placeholder}
  \caption{Axis Shadow Rendering}
  \label{fig:axis-shadow}
\end{figure}

Simple right? Now, here is how we get from this silly looking grid to the good stuff. Each intersection of two
lines is a tile. The shape of the tile is governed by the geometry of the neighboring points. Here is an example
with tiles shrunk for clarity:

\begin{figure}[h]
  \centering
  \includegraphics[width=0.8\linewidth]{figures/placeholder}
  \caption{Rendering of tiles seen by use of de Bruijn's method.}
  \label{fig:db-vis}
\end{figure}

The only caveat is what do we do if more than two lines intersect at one place? Well, simply put we don’t let that happen in order to accomplish this all we have to do is shift the origin of the coordinate system in a way to make that impossible. Recall we are living in an isometric space, this means to avoid collision we need to make sure our meters are mutually non interchangable. This is simpler than it sounds, here is an example:
\begin{equation}
 \left[1, \sqrt{2}, \sqrt{3}, \sqrt{5}, \sqrt{7}\right] \mathrm{TODO PR_1}
 \label{eq:pr_1}
\end{equation}

\begin{equation}
 \left[1, \sqrt{2}, \sqrt{3}, \sqrt{5}, \sqrt{7}\right] \mathrm{TODO PR_2}
 \label{eq:pr_2}
\end{equation}

\begin{equation}
 \left[1, \sqrt{2}, \sqrt{3}, \sqrt{5}, \sqrt{7}\right] \mathrm{TODO PR_3}
 \label{eq:pr_3}
\end{equation}

There is no way to multiply any of these by integer quantaties and arrive at the same number. Thats all there is to it! Lets take a break and look at some tilings:

\section{Results}

Here is where the fun part comes in. struck by the similarities and differences of these tilings, and the people I
associate with them, I wondered what it would be like to be some sort of intermediate tiling, and if such a thing 
were well defined. It turns out, it is! A good chunk of the new discovery I have done in this space boils down to
the fact that for most pairs of Penrose Tilings there exist an intermediate form which is also a Penrose tiling.
This averaging process holds, and allows us to form further intermediates up to a continuum of tilings. Neat! 

I set about trying to visualize this in some form. My initial work was to make videos where subsequent time points were
the result of moving between two tilings. You can seem some of them TODO Link. While this is an easy approach, I didn’t
find the results particularly satisfying. Fundamentally we are adding a dimension to the visualization, which feels like
cheating.


\begin{figure}[h]
  \centering
  \includegraphics[width=0.8\linewidth]{figures/placeholder}
  \caption{Super-Quasicrystal tiling}
  \label{fig:gm-tiling}
\end{figure}

Here are some neat things about this tiling that I am working to prove:
\begin{enumerate}
	\item Every single tile is unique, that is to say in addition to there being all the same trappings of irrational character, the individual tiles themselves embody this character.
	\item It has only S1 symmetry. That is to say, there is no way in which it can be rotated or reflected to see itself. Penrose tilings feature a single D5 symmetry point meaning that the center can be rotated/reflected in 5 ways.
	\item Each vector that starts from the center represents a Penrose tiling. That is to say that rolled into this figure is an infinite number of tilings and an analagous way to Mandlebrot/Julia-set pairings. This diagram contains all three canonical Penrose Tilings in a compressed form.
\end{enumerate}

\bibliography{paper.bib}
\end{document}